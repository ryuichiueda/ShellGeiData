\begin{align}
	\Delta \V{x}_t \sim p( \Delta\V{x} | \V{x}_{t-1}, \V{u}_t) \label{eq:state_transition_model}
\end{align}
と表わすことができます。また,移動量ではなく,
移動後の位置$\V{x}_t = \Delta \V{x}_t + \V{x}_{t-1}$の分布で考えると,
\begin{align}
	\V{x}_t \sim p( \V{x} | \V{x}_{t-1}, \V{u}_t) \label{eq:state_transition_model2}
\end{align}
と表せます
\footnote{
	$p$は式\ref{eq:state_transition_model}とは異なる関数になりますが,
	同じ記号で表すので文脈で頭を切り替えてください。
}。
実験環境を整えたりすることになります。

これらの式を図にしたものを\ref{fig:motion}に示します。

\begin{figure}[hbt]
 \centering
  \includegraphics[width=0.6\linewidth]{figures/motion.eps}
	\caption{$\V{u}_t$と分布$p(\V{x} | \V{x}_{t-1}, \V{u}_t)$}
  \label{fig:motion}
\end{figure}

式\ref{eq:state_transition_model},
\ref{eq:state_transition_model2}の
右辺の分布については,「条件に書いていないこと」も重要です。
これらの分布では,
「$\V{x}_{t-2}$以前のロボットの位置は,$\Delta \V{x}_t$や$\V{x}_t$の
値に影響を与えないと考える」
と暗に宣言していることになります。
これは,ロボットが高速で移動していると成り立ちませんが,
低速で動いていればだいたい成り立ちます。


式\ref{eq:state_transition_model}や
式\ref{eq:state_transition_model2}の表現は,
特にロボットだけでなく,動くものならなんでも
適用できる式です。
たとえば\ref{fig:typhoon}の台風は
特に意思を持っていませんが,
周辺の風や,その他台風の動きに影響のあるものを
すべて$\V{u}$のなかに変数として押し込めることで,
式\ref{eq:state_transition_model2}のように表現できます。
また,制御の教科書を開くと,
\begin{align}
	\V{x}_t = A \V{x}_{t-1} + B \V{u}_t + \V{\varepsilon}\label{eq:state_equation_linear}
\end{align}
($A$,$B$は行列,$\V{\varepsilon}$は雑音を表すベクトル)
や,
\begin{align}
	\V{x}_t = \V{f}( \V{x}_{t-1}, \V{u}_t) + \V{\varepsilon} \label{eq:state_equation_nonlinear}
\end{align}
という式(\textbf{状態方程式}\index{じょうたいほうていしき@状態方程式})
が最初に書いてあります。これらの式と,
式\ref{eq:state_transition_model2}を見比べると,
式\ref{eq:state_transition_model2}が,
これらの式を抽象化したものであることが分かります。
つまり,制御で扱われる物体の動きは,
式\ref{eq:state_transition_model2}で表現できるということになります。
